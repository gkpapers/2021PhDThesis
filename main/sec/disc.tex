The work presented across the manuscripts contained within this thesis demonstrate the creation and application of
infrastructure and methodology for the analysis of stability in neuroimaging pipelines, and demonstrate its
application for both the study of robustness in an analytical context and the generation of more generalizable
brain-phenotype relationships. The four component manuscripts incrementally present the methods developed to
perform a large scale evaluation of numerical stability in neuroimaging, each providing a foundation for the
following studies. As a result, this thesis has led to the creation of both scientific software resources and
considerable scientific advancement in the study of numerical stability in neuroimaging.

In the following sections I will discuss and interpret the findings as a whole and their implications on the field
of structural connectomics. I will then discuss the relevance of extrapolating these conclusions to other domains
of neuroimaging, including the analysis of both functional and structural images. Beyond discussing the
implications of the results presented here, I will propose ways in which perturbation analyses may be further
adopted in neuroimaging, and conclude with lessons learned both from my own pursuit of these projects and
recommendations for the field at large.

\subsection{The Impact of Perturbing Structural Connectome Estimation}

Throughout this thesis I have demonstrated that controlled perturbations serve as a viable and effective method
to study the stability of structural connectome estimation pipelines. In each of the various methods tested in
Chapter II, such as one-voxel noise injection, sparse Python-only Monte Carlo Arithmetic (MCA), or dense Full-stack
MCA, we noticed that the distinct nature of the perturbations in each case led to distinct variance in the results.
In the case of one-voxel perturbations, a small and localized form of noise, we noticed similarly localized changes
in outputs. However, the MCA instrumentations led to outcomes with local, topographical, or scaling changes across
different simulations of the same data. Perturbations with MCA were not only able to encapsulate similar effects
and deviations as were observed with the one-voxel methods, but the independence of the MCA method from the dataset,
the lack of introduced bias being introduced, and the global perturbation of operations makes MCA a more
scalable and generalizable technique. While I recommend the adoption of the MCA technique generally, one-voxel
methods can serve as a valuable method for the local evaluation of instabilities, such as an evaluation of the
sensitivity of an algorithm in a region that is difficult model, or in proximity to unavoidable contrast changes
such as in the presence of a lesion or significant atrophy.

The two MCA environments used throughout this thesis differed considerably in both their construction and the
resulting behaviour of tools. In the sparse setting, referred to across the manuscripts as either ``Python-only''
or ``Input Instrumentation'', a considerably smaller set of libraries were instrumented with MCA. In this case,
all floating point operations performed directly by Python or Cython-compiled code were perturbed, however, while
the widely adopted NumPy~\cite{harris2020array} library is a Python-installed module, it is largely
uninstrumented as operations therein predominantly rely on lower-level libraries BLAS~\cite{lawson1979basic} and
LAPACK~\cite{anderson1999lapack}. In the dense setting, referred to as ``Full-Stack'' or ``Pipeline
Instrumentation'', BLAS, LAPACK, and NumPy were instrumented as well. This distinction meant that the operations
perturbed in the sparse configuration were not those doing the ``heavy lifting'' of image, matrix, or array
manipulation, but only those doing more simple manipulations of data in between these operations. This results in
an important change to the MCA method as it was originally defined by Parker~\cite{Parker1997-qq}: the
few-and-far-between nature of perturbations both limits the law of large numbers~\cite{hsu1947complete} and
allows for the cascading of single perturbations without correction, and it is thus likely that there exists
numerical bias in perturbations introduced in the spare implementation as a result. The realizations of this bias,
however, is random across executions, such that an unbiased sampling of the distribution of results may still be
possible.

The observation that the bounds of variability were similar across the two MCA instrumentations, alongside
considerably reduced computational overhead in the Python-only case, suggests that sparsely applied MCA could be a
viable technique for approximating the bounds of pipeline variation as measured with dense (or, true) MCA. Just as
the list of instrumented libraries in the sparse configuration is a strict subset of those instrumented in the
dense case, it holds that the set of possible perturbed results in the sparse configuration is a strict subset of
the possible results in the dense case. This relationships is potentially powerful as this means that problems can
undergo a preliminary and less computationally intense stability evaluation prior to being evaluated more densely.
The coherence of results across the two perturbation configurations presented throughout Chapter III suggest that
the sparse Input Instrumentation is a high quality estimator of the true MCA distribution. Figure~\ref{ch3f2}
shows that the sample distributions of network statistics derived from the perturbed connectomes are unchanged
across both MCA configurations, and Table~\ref{ch3t1} demonstrates the ability of both techniques to increase the
reliability of the dataset. Uniquely, the connectomes resulting from the sparse MCA instrumentation contained less
session- and acquisition-dependent signal than either the original dataset or the true MCA results. While the
specific cause of this is result unknown, it is impossible to make a statement on the generalizability of the
effect, however, given its potential significance in brain imaging, it is an area of extreme interest for future
work.

The variability introduced by MCA also had considerable impact on the modelling of brain-phenotype relationships.
In the case of both MCA instrumentation methods, Figure~\ref{ch3f3} shows how random sampling of the perturbed
networks led to a distribution of performance (F1 score) on a BMI classification task spanning an approximate range
of $\pm 0.10$ relative to reference. The balance of performance about the original values lends credence to the
interpretation of MCA derived connectomes as being sampled from the distribution of equally plausible results,
some of which may contain ``more'' or ``less'' signal for a given task, and the associated unperturbed connectomes
are simply point-estimates of these distributions. Chapter IV demonstrates several techniques for considering
these distributions on a similar classification task, including aggregation strategies which aim to capture the
added variability and a truncation strategy which removes it. When exploring these techniques, the truncation
strategy (which only retained unchanging bits for all observed intensities within a connectome) was the only method
which did not improve the performance of the models, suggesting that the observed variability doesn't amount
entirely to random fluctuation or noise, but that the variation observed across MCA evaluations contains meaningful
signal. Each of the aggregation strategies led to improved performance of the classifiers, while three strategies
led to the improved generalizability of these classifiers as well (e.g. similarity between performance on validation
and test sets): distance-dependent consensus averaging~\cite{Betzel2018-eo}, mega-analytic consideration of all
connectomes, and meta-analytic voting across classifiers trained on jackknife sampled datasets. The thing that each
of these techniques have in common over the other methods tested (independent jackknife classifiers, median, mean),
is that more of the variance observed across perturbations is captured rather than sampled or smoothed. The
variance across perturbations isn't captured in the case of each jackknife classifier, it is just uniquely sampled,
while median and mean consider the variance uniformly and without contextual importance, which introduces biases in
structure that motivated the creation of the distance-dependent consensus technique~\cite{Betzel2018-eo}.

The combination of results presented across my thesis, showing both the significance of numerical instabilities in
structural connectome estimation and the utility of the associated variability, demonstrate that numerical analysis
techniques such as MCA can serve as valuable tools towards studying and improving reproducibility in neuroimaging.
Other efforts which have explored reproducibility from the angle of analytical
flexibility~\cite{botvinik2020variability,schilling2020tractograph} sample a broad and rich space of approaches and
tool configurations that is not directly comparable to the techniques demonstrated here. Combining these approaches
would be of extreme interest and lead to a far richer description of the uncertainty associated with certain tasks,
processing techniques, or analytic questions, and could be used to inform comparison of results obtained across
software libraries.

While I have yet to mention the Clowdr tool presented in Chapter I, it played an essential role in the execution of
these experiments. Over the course of the four presented manuscripts, Clowdr has facilitated the launch, debugging,
visualization, and provenance recording of over $20,000$ cluster and cloud tasks, totalling $22$ CPU-years worth of
resources. While many tools exist to facilitate large scale deployments of pipelines, Clowdr enables this for tools
which are still in development while also providing a quick path to the FAIR~\cite{wilkinson2016fair} publication of
tools and records through Boutiques~\cite{Glatard2018-tu}.

\subsection{Extrapolating Conclusions Across Domains}

Throughout this thesis I have focused on the analysis of stability in the estimation of diffusion MRI-derived
structural connectomes. Given this specific application, I do not make any strict claims about the generalizability
of these findings on other domains of neuroimaging or beyond. However, if one wishes to participate in the thought
experiment of forming a hypothesis about the stability of another domain, there are three components which must
be considered: data, problem difficulty, and software quality. I will discuss what to look for in each of these
components, and apply these questions to possible applications in structural and functional neuroimaging.

\paragraph*{Data}
The quality and dimensionality of datasets, summarized through features such as the signal to noise ratio (SNR),
resolution, sparsity, and image contrast all play a role in the stability of operations performed on this data. The
impact of each of these features can be considered through fixing the others and considering the implications in the
context of the curse of dimensionality~\cite{friedman1997bias}. For example, for a setting in which equivalent
datasets vary only by the sparsity of entries, applied models will arrive at more stable solutions in the case of
the more dense dataset~\cite{geman1992neural}. Similarly, if a dataset has a higher SNR relative to an otherwise
equivalent neighbour, the dataset with the higher SNR will lead to more stable solutions. In the context of
neuroimaging, the resolution and sparsity are relatively consistent across all MRI modalities, however, the SNR and
quality of contrasts differ significantly. Structural imaging methods generally have higher SNR and sharper
contrast~\cite{bergamino2014review,chavhan2009principles} than either functional~\cite{logothetis2004nature} or
diffusion~\cite{thomason2011diffusion} sequences, suggesting that the analysis of structural images may lead to more
robust results. However, if analysis methods have been developed with underlying assumptions about the SNR or
smoothness of these images, then the associated tools may in fact be less stable in the face of perturbation.

\paragraph*{Problem Difficulty}
The difficulty of a given problem can also be thought of as the model's complexity. The theoretical difficulty of
robustly fitting a model can be evaluated by calculating its conditioning, as was discussed in
Section~\ref{sec:bg_stab}. While this remains practically difficult to evaluate, a reality which served as a large
motivation for this thesis, complexity can be coarsely estimated by considering both the dimensionality of input
data and the number of parameters produced in the output. For example, a diffusion MRI dataset containing $35$
diffusion directions will provide a more stable solution when fitting a $6$ component tensor than an $60$ component
orientation histogram, barring any fatal algorithmic flaws in the former. In reality, the difficulty of a pipeline
is dependent on the specific steps, their sequence, and the specific algorithms chosen to accomplish them. While
there are many stable diffusion tensor models~\cite{skare2000condition}, it is possible that this area provides a
best-case given that each voxel is modelled independently and the models themselves summarize relatively simple
phenomenon. On the other hand, techniques for the correction of topological errors in brain surfaces, a common
component of both structural and functional analyses, have been developed over several decades and range in
complexity from manual proof-reading to fitting high dimensional transformations~\cite{yotter2011topological}. It
is likely that there is considerable heterogeneity in the stability of these methods given the dramatic range in
flexibility, complexity, and conceptual approach. Similar to image registration, simpler (e.g. affine) approaches
will tend to be considerably more stable than their more complex (e.g. diffeomorphic non-linear) counterparts.
This balance between complexity or problem difficulty and stability is therefore a design decision which must be
considered when constructing pipelines and, absent a context-specific evaluation, simpler methods should be
preferred if variability in the results is undesired.


\paragraph*{Software Quality}
There are many factors which determine the quality of software, though without explicit evaluation these can often
best be approximated by meta-data properties of libraries. One such feature is the openness of a tool. Open science
practices have recently been shown to lead to faster progress~\cite{munafo2017manifesto} and have been widely and
effectively been adopted in both genomics~\cite{goecks2010galaxy} and for the ongoing COVID-19
pandemic~\cite{besanccon2020open}. Aside from the ``two heads are better than one'' argument, this is likely
because software quality has been shown to decrease in a competitive market~\cite{raghunathan2005open}, while open
source software development fosters a collaborative environment. Another feature of library quality is modularity
or size; smaller libraries solving fewer independent problems are generally of higher
quality~\cite{raghunathan2005open}. With these criteria in mind, the diffusion workflows evaluated here relied
solely on open source implementations of thoroughly tested and peer-review published
algorithms in DIPY~\cite{Garyfallidis2014-ql,Garyfallidis2012-gg}, which makes the quality of software a likely
best-case scenario. There exists considerable heterogeneity in both the openness and size of libraries across
neuroimaging. While pipelines such as fMRIprep~\cite{esteban2019fmriprep} integrate functionality across a set of
disparate libraries, it is likely that distinct branches therein will differ considerably in terms of their
stability. Both FreeSurfer~\cite{fischl2012freesurfer} and CIVET~\cite{lepage2017human}, two tools commonly used
for the estimation of cortical surfaces, have been largely developed in separate silos. While these tools are more
mature than either DIPY or fMRIprep mentioned above, it has been shown that this maturity does not equate to a
robustness to minor data perturbations~\cite{Lewis2017-ll} in either case. The estimation of software quality is
undeniably the most difficult to perform from a surface level, making it imprudent to provide a general claim as
to the expectation of which domains or analytic modalities may be more stable relative to others.

While this exercise could be conducted across arbitrary domains, if one had to make an \textit{a priori} set of
guidelines to follow it would unsurprisingly be that higher quality data, simpler problems, and more open and
widely tested code will generally lead to more stable solutions\footnote{This recommendation towards simplicity
is not unlike the fact that simple or heavily regularized models are more generalizable in machine
learning~\cite{lever2016regularization}.}.


\subsection{Future Directions in Operationalizing Perturbation Analysis}
The initial aim of this thesis was to identify instabilities within neuroimaging pipelines. An exciting evolution
of this plan was the finding that the variability arising from numerical perturbations may contain meaningful signal.
This suggests that perturbation analysis is not only valuable as a tool for measuring the stability of algorithms or
workflows, but that it can be applied far more flexibly to improve their quality. Considering this, next steps in
this area could be categorized into two broad questions: \textit{where} to perturb, and \textit{what} to do with the
induced variability. I will discuss different components of neuroimaging workflows which may benefit from perturbation
analysis and identify several possible applications of the results for each. I will then more broadly discuss other
applications of perturbation analysis as it may apply across each of these areas. Given that the intersection of
numerical analysis and neuroimaging is in its infancy, many of these potential explorations remain as open questions
that an interested reader could explore (or build a grant program around).

\subsubsection{Targets for Perturbation}

Beginning with image reconstruction, initiatives such as Gadgetron~\cite{hansen2013gadgetron} provide
scanner-agnostic MRI reconstruction algorithms in an effort to both improve the quality of these reconstructions
and reduce cross-vendor variances. The perturbation of image reconstruction could be performed to identify which
algorithms are the most robust across a set of manufacturers and scanner models. In addition to the identification of
robust reconstruction methods, higher fidelity images could be generated through the ensemble of perturbed images,
possibly achieving benefits like those found in High Dynamic Range image enhancement. Similarly, this could be applied
to influence the development of vendor-neutral pulse sequences themselves~\cite{karakuzu2020qmrlab}, and inform the
adoption of stable data collection and reconstruction pairs.

Reconstructed images typically undergo numerous alignments, corrections, denoising steps, and modelling prior to their
application in experiments. These processing steps, such as image registration which is relevant across all modalities
of neuroimaging, are often known to exhibit sensitivity to initial conditions or noise~\cite{salari2020file}.
Perturbing any of these components in a workflow is similar to what was done in this thesis, where the results could
be used to test the sensitivity of a given component to numerical error, identify unstable steps within pipelines, or
be aggregated to form more stable solutions.

Processed data are often terminally used in statistical testing or machine learning frameworks. Commonly used
frameworks for testing hypotheses or fitting models often contain an element of randomness, such as Permutation
Testing~\cite{oden1975arguments} or Stochastic Gradient Descent~\cite{bottou1991stochastic}, likely making them less
susceptible minor perturbation. However, feature selection, a typically deterministic process, is commonly performed
on neuroimaging data prior to their application in these frameworks~\cite{mwangi2014review}. In cases where
initialization of these features has considerable impact on their quality~\cite{kobak2021init}, the perturbation of
both the initializations and feature selection process serves as an interesting avenue for exploration. It is possible
that perturbing feature selection will lead to similarly meaningful variability as was observed here, which would
have a considerable benefit of being much less computationally expensive than the perturbation of image processing
steps.

\subsubsection{Applications of Perturbations}

This thesis ultimately focused on two specific applications and interpretations of perturbation analyses: the
evaluation of pipeline stability and the aggregation of unstable derivatives. There are many possible extensions to
this work and distinct functions that MCA-based perturbations can enable from both the perspective of tool
development and analysis. One such opportunity from the perspective of tool development is the localization of
instabilities. VeriTracer~\cite{chatelain2018veritracer} provides functionality to trace the execution of
workflows and evaluate the propagation of numerical error throughout pipelines. By gradually improving or
removing sources of instability within pipelines, it would be possible to construct workflows that are more
robust to numerical error. A separate approach would be to introduce a mixed (declining) precision model for
data in pipelines. In this case, MCA could be used to estimate the number of significant bits at each node of
a pipeline, and use this information to truncate the bit-space allotted to subsequent results accordingly. This
approach would not only remove the compounding effect of numerical instabilities at each node (defined at a
context- and tool-appropriate resolution), but it would make explicit a priority for tool developers to err on
the side of building simpler and more robust pipelines. In a prelimiary exploration of this idea, I truncated the
data associated with Chapter III after pre-processing was performed and found that a large portion of the induced
variance was reduced in the input instrumentation case.

In cases where the re-engineering of pipelines is either non-feasible (e.g. closed source software) or not of
interest, I demonstrated that capturing the induced variability can improve the quality of derived results. An
interesting comparison can be made here between pipeline variability and the Bias-Variance trade
off~\cite{jain2000statistical}. Deterministic pipelines without perturbations will produce a single result with a
single fixed (and possibly biased) error. When perturbations are introduced, the variability in this error becomes
apparent and can be modelled, shifting the trade off in favour of variance. Rather than then considering any single
result of the tool as the ``true'' result, we can model a series of results derived from the same data in the form
of a \textit{mean} $\pm$ \textit{variance}. This shift in perspective is powerful: statistics have long concerned
themselves with accounting for measurement error in the collection of samples and developing robust measure for
estimating population-level statistics. Adopting such techniques across perturbed observations will not only allow
for more richly descriptive individual measures, but it will provide a quantitative value of measurement uncertainty
which can be accounted for in subsequent testing and modelling efforts. This would enable the construction of a
joint false-positive rate when combining the uncertainty in results and testing frameworks, and would lead to more
reproducible findings.

\subsection{20/20}
The final topic I would like to discuss is centered on hindsight. Throughout this project I have developed tools
for scientific infrastructure, built pipelines from modular pre-existing components, adapted a numerical analysis
framework to a new scientific domain, modelled instability in connectomes through a wide array of commonly used
and novel measures, and showed how numerical instability can be a useful property of software. This process has not
only been punctuated by scientific outputs which found their ways into papers, but by failures and lessons, both
pertaining to my own work and the field of neuroimaging at large. I have published successes and failures from my
Ph.D. on my website throughout my degree\footnote{\url{https://gkiar.me/phd}} in an effort to not repeat my own
mistakes and allow others to learn from them. While making an effort to be brief, I feel it necessary to provide
some considerations and advice that would be relevant both to my previous-self and possibly the field of
neuroimaging as a whole. None of these reflections are intended to be perfect, but merely serve as some things to
think about...

\paragraph*{Be Aware of the Cost of Computing}
While this advice may at first provoke thought about funding and the cost of either building machines or using the
cloud, I do not intend to refer to money, but energy. Scientific computing is an expensive endeavour: over the course
of my thesis, I have consumed nearly $22$ CPU-years worth of execution time for the project directly related to my
thesis, and double that if considering collaborations. Even taking a lower-bound estimate, this means my experiments
have consumed $2,000$ kWh of electricity\footnote{\url{https://www.top500.org/lists/green500/2019/11/}}. I have been
fortunate to perform my research on computers residing in Quebec and Ontario, where the rate of Carbon emissions at
the time of writing peaks at approximatley $20$ g CO\textsubscript{2}/kWh\footnote{\url{http://bit.ly/canada-energy-consumption}}.
In this relatively green environment my research has produced $80$ kg CO\textsubscript{2} emissions which, for
reference, currently equates to an upper-bounded offset cost of approximately $\$4.00$\footnote{\url{https://www.energysage.com/other-clean-options/carbon-offsets/costs-and-benefits-carbon-offsets/}}.
If I had performed my experiments in another environment, such as through Amazon Web Services instances residing in
the United States where the rate of Carbon emissions averages over $400$ g CO\textsuperscript{2}/kWh\footnote{\url{https://www.eia.gov/tools/faqs/faq.php?id=74}},
the total environmental footprint would increase $20$-fold, which would amount to the same Carbon footprint as putting
and extra car on the road for half of a year\footnote{\url{https://www.nrcan.gc.ca/sites/www.nrcan.gc.ca/files/oee/pdf/transportation/fuel-efficient-technologies/autosmart_factsheet_6_e.pdf}}!
This is not meant as an endorsement for running analyses in Canada, but merely as a reminder that we should be aware
of the impact that our experiments have on the world more broadly than just our papers. This is important so we can
clearly evaluate which experiments are worth doing and offset any distant harm we may cause.

\paragraph*{Keep Clear and Incremental Objectives}
Over the last several years it has become clear to me that projects are rarely ``complete''. There is always
another experiment, figure, analysis, or interpretation that we can explore. However, the longer we follow these
rabbit holes (which often provide their fair share of excitement), I have found it becomes gradually easier to lose
sight of the original target and purpose. The experiments I presented in Chapter~IV, for example, are a small
subset of the models, conditions, and experimental permutations that I originally performed when aggregating
connectomes. I did not remove the others because of their outcomes, but because they were the result of iteratively
asking refining my experiment that brought me gradually further from answering the question I sought out to explore.
This led me into the territory of having answers that may have been interesting, but that were ultimately
uninterpretable because I hadn't answered a question that they depended upon. I suspect that I am not the only
scientist to fall into this trap, and I believe the field of neuroimaging as a whole may have experience with it, at
least in appearance. Despite regularly defining possible real-world applications and outcomes in papers, neuroimaging
has received considerable criticism for lack of delivering on this promise~\cite{robinson2004fmri}. While this
criticism is perhaps unfairly harsh~\cite{lyon2017dead}, it remains that neuroimaging hasn't become as relevant as
may have once been anticipated. While I believe this is in part due to an academic curiosity taking the reins, I
think this in reality is an ``image problem'' due to conflation of the \textit{eventual} objectives and those which
are presently feasible and being built towards. By clearly defining our goals and the incremental objectives we must
meet along the way, both as individual scientists and as a field, I suspect we'll fall down fewer of these rabbit
holes, and stop having to defend the reality that science is an incremental process that takes time.

\paragraph*{Avoid ``Not Invented Here'' Thinking}
A point which I have mentioned repeatedly throughout this thesis is that there are a considerable number of
libraries that have been built to accomplish similar tasks, often in isolation. In fact, I committed this taboo
myself during this thesis in Chapter~I with the creation of Clowdr. I built this tool out of the necessity to run
experiments in the particular way I had envisioned, and I did so largely from the ground-up. As a result, the code
has at various points required development to function in new use cases, or become difficult to maintain as its sole
core contributor. In retrospect, a more sustainable solution may have been for me to build my extension into an
existing framework or library, such as Nipype~\cite{gorgolewski2011nipype}. While the creation of this toool was
necessary and relevant for the completion of my thesis, if I built the extensions I needed in an ecosystem I would
have not only gained the review and debugging support of its user and developer base, but also been able to provide a
more accessible contribution to the community who were already familiar with a given environment. I believe the same
could likely be said for many tools that have been built in neuroimaging (and likely science more broadly). Though I
don't believe there should be a single solution for every challenge, I do think that maintaining a collaborate-first
and compete-second perspective would not only reduce possible redundancies in this space, but would lead to higher
quality and more sustainable contributions.
