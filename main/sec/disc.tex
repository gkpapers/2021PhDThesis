High level summary of the work and break-down of the following sections

\subsection{The Sum of its Parts}
what this thesis ultimately amounted to and taught us

describing the variability observed and its impact

comparing and contrasting to other sources of variability 

\subsection{Extrapolating Conclusions}
Discussing the likelihood that these findings hold across pipelines, modalities, or other scientific disciplines in
general.

Perhaps approach it by first assuming that this is a worst-case discipline, and that everything happening here is
especially bad, and and walk along that logic by iteratively disproving that this field is worse than others, so it is
likely that while specific details of the findings will change, it is far more likely that this is the case in other
domains than that it is not

\subsection{Future Directions in Operationalizing Stability}
brief reminder of the fact that it isn't only about what we can perturb but why we can perturb

\subsubsection{Where to instrument}
acquisition

other parts of the image processing pipeline

other modalities

machine learning models, in particular feature selection or other typically deterministic components of the process.

\subsubsection{How to use it}
localizing instabilities within pipelines and correcting them

Mixed (declining precision models) for pipeline development, such that the stored precision never exceeds the number of
significant digits throughout the evolution of a pipeline.

Aggregation across perturbations as the new ``mean $\pm$ variance''

Development of more robust and rich biomarkers including capture of distributions of variability as a new dimension

Coming up with error bars of results, not just error bars on statistical tests performed on those results; subsequently
creating a joint false-positive rate as a combined function of the two

\subsection{20/20}
What is, ``hindsight'', Greg. 

all of these lessons are both for myself and for the community

These recommendations may be in conflict with what the academic incentive structure may encourage, or contrary to what
the ``correct'' decisions may have been when neuroimaging originated. They are not intended to be perfect, merely
lessons that I feel are worth bearing in mind if one were to go start over today.

collaborate, don't compete. build tools as part of supported ecosystems, not in isolation. ``two heads are better
than one'', and the variability of bandwidth overtime makes tools going stale a challenging problem to stay ahead of
unless you're part of a team.

test. this hardly needs to be elaborated upon, but nothing works until you can prove it works

walk, don't run. While it may be fun to sprint through an experiment or field of study, claiming success when we get to
the finish line, it is far more likely that we lead a pack of our colleagues to a dead end than if we slowed down and
checked the map after each step.

Be humble. Be open about our strengths and weaknesses, and collaborate so that we not only all get to spend the most
time using our strength, but we can learn from others around us. Science by definition is interdisciplinary, so we
should admit this and let ourselves shine where we're experts while encouraging others to be the same, collectively
learning and building towards something bigger.

don't become emotionally attached. Too many mistakes happen to loyalty to an idea or bias in interpretting our results
towards an expected outcome. We are human, this is normal, but this is ultimately to our own dismay. Be objective when
you can, and if you can't, call somebody who can be or who expects the opposite result from you so that you can learn
from each other.

Define the objective, and keep it in mind. Too often we fall down rabbit holes and lose touch with the reality of what
we're trying to accomplish. If the goal is academic, define milestones of understanding. If the goal is practical,
define measure of success or failure and timelines for accomplishing them. It becomes far too easy to get swept away in
it all as we pick up momentum, and that does a disservice to ourselves, our funders, and by extension our societies.

