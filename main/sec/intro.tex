\begin{itemize}
\item neuroimaging seeks to relate human brains to behaviours
\item this often relies on mri
\item is the the only in-vivo non-invasive technology that allows high-resolution evaluation of brain structure in humans
\item the images are processed by complex phenomenological modeling software, and statistical models applied on the derivatives
\item however, given the complexity of methods, reproducibility can be an issue
\item reproducibility stems from a number of levels, ranging from re-executability to numerical instabilities
\item this thesis seeks to:
\begin{itemize}
\item provide a solution to the re-executability of neuromaging pipelines
\item explore methods for evaluating instabilities in neuroimaging pipelines
\item quantify the analytic impact of induced instabilities
\item leverage numerical instabilities towards improving the quality of neuroimaging research
\end{itemize}
\item ultimately, I created a piece of computational infrastructure which was used to facilitate the execution of pipelines consuming millions of cpu-hours, effectively induced instabilities in numerical pipelines through Monte Carlo Arithmetic, described the significant impact of these instabilities in analytic contexts, and proposed a method for leveraging instabilities in machine learning contexts which improves the generalizability of learned models.
\item my thesis not only sheds light on an impactful issue in neuroimaging, but it demonstrates a method for both shedding light on and benefitting from this issue by researchers in practice.
\end{itemize}
