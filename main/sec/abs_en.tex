Neuroimaging, a field focused on understanding structure and function within the brain, has evolved into a
discipline of computational science. In recent years, through the collection and evaluation of rich and
descriptive datasets, countless studies have explored the possible utility of Magnetic Resonance Imaging data
of the brain and began the development of relationships between them and phenotypic information such as
disease, sex, stage of development, and more. Like many modern scientific disciplines, however, the
reliability of these relationships depends on the trustworthiness and openness of methods used in their
generation. With an increasing awareness on issues of reproducibility, preliminary explorations in
neuroimaging have led to some skepticism regarding the robustness of findings. In some cases, these issues
can be resolved through the federated sharing of datasets or the availability of software tools. In others,
the sensitivity or stability of underlying numerical algorithms used to process the complex and high
dimensional datasets may lead to unexpected variability. While impactful differences have been observed in
neuroimaging across scientific teams, methods, toolboxes, or even the operating system of the computer being
used, no studies have enabled or characterized the effect that machine error has on findings. Throughout
this thesis I have developed infrastructure for high performance computing environments to enable the large
scale numerical analysis of neuroimaging experiments, and used it to evaluate the role that numerical
uncertainty in computing plays in the reproducibility or stability of results. Focusing on the application of
mapping connections between brain regions, called structural connectomics, I perturbed the generation of
networks and subsequently evaluated the observed variability to characterize the impact of small numerical
errors as they accumulate over the course of an analysis. I found that not only do unavoidable computational
errors have a considerable impact on results, but that they can in some cases be so significant as to obscure
individual differences in maps of connectivity. This finding empirically shows the minimal/ground error
induced and propagated by the floating-point model throughout scientific pipelines. Moving beyond the
perturbation of error, efforts to interpret the observed variability as a \textit{feature} of a given analysis
rather than a \textit{flaw} showed that not only could this variability be used increase the reliability of
datasets, but it could be counter-intuitively harnessed to produce stronger and more generalizable
brain-phenotype relationships. The combination of these results importantly demonstrates that computational
errors inherent to pipelines lead to impactful differences in a set of neuroimaging analyses. Whether that
impact is for better or worse, however, depends entirely on whether these errors are being considered and
accounted for, or ignored. Alongside the execution of these analyses, this thesis presents tools and a
schematic which can be followed for the further exploration of numerical uncertainty and the role it may play
in our understanding of the brain.

