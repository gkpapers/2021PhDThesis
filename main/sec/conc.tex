The objective of this thesis was to explore and improve the robustness of results in neuroimaging. This process
began with the development of a scalable library for the deployment, management, visualization, and
provenance-rich capture of computational experiments, which significantly lowered the barrier to adopting high
performance computing infrastructures for large neuroimaging experiments. With the computational backbone in
place, I compared various perturbation methods for the evaluation of numerical instabilities in structural
connectome estimation pipelines. Through the perturbation of either the inputs or adding stochasticity to the
inevitable numerical error, I was able to demonstrate instability in structural connectome estimation that peaked
at the level of observing cross-subject effects. Building upon this proof-of-concept for the effective perturbation
of pipelines, I performed a series of realistic experiments, comparisons, and evaluaions of perturbed structural
connectomes and quantified the impact of the observed instabilities. While individual estimates of connectivity and
subsequently derived features varied considerably, in some cases retaining no signal, it became evident that the
perturbation of connectomes led to a new source of variability that both improved the reliability of the dataset and 
had potential for improving the downstream modelling of brain-phenotype relationships. In an effort to capture this
variance, I devised and tested several strategies for the aggregation of this variance across perturbed connectomes
and evaluated them by proxy of performance in a machine learning context, and found that the aggregation of
variability lead to more generalizable and performant predictive models.

In summary, I believe that my thesis clearly demonstrates the impact that numerical instabilities may have on
neuroimaging results, and provided a compelling solution to this problem that can be immediately adopted by the
community. Alongside the characterization of instability in structural connectome estimation, the tools I developed
increase the accessibility of large scale applications of numerical analysis and neuroimaging. I began this thesis
with the ambition of shedding light on important issues in neuroscience, and ultimately improving the trustworthiness
of neuroimaging. I believe that the contributions presented here provide compelling evidence of both the potential
impact of numerical error in neuroimaging, and provide a glimpse of a hopeful future in the field when these
challenges are faced head-on rather than swept under the rug.

